%% Generated by Sphinx.
\def\sphinxdocclass{report}
\documentclass[letterpaper,10pt,english]{sphinxmanual}
\ifdefined\pdfpxdimen
   \let\sphinxpxdimen\pdfpxdimen\else\newdimen\sphinxpxdimen
\fi \sphinxpxdimen=.75bp\relax
\ifdefined\pdfimageresolution
    \pdfimageresolution= \numexpr \dimexpr1in\relax/\sphinxpxdimen\relax
\fi
%% let collapsible pdf bookmarks panel have high depth per default
\PassOptionsToPackage{bookmarksdepth=5}{hyperref}

\PassOptionsToPackage{booktabs}{sphinx}
\PassOptionsToPackage{colorrows}{sphinx}

\PassOptionsToPackage{warn}{textcomp}
\usepackage[utf8]{inputenc}
\ifdefined\DeclareUnicodeCharacter
% support both utf8 and utf8x syntaxes
  \ifdefined\DeclareUnicodeCharacterAsOptional
    \def\sphinxDUC#1{\DeclareUnicodeCharacter{"#1}}
  \else
    \let\sphinxDUC\DeclareUnicodeCharacter
  \fi
  \sphinxDUC{00A0}{\nobreakspace}
  \sphinxDUC{2500}{\sphinxunichar{2500}}
  \sphinxDUC{2502}{\sphinxunichar{2502}}
  \sphinxDUC{2514}{\sphinxunichar{2514}}
  \sphinxDUC{251C}{\sphinxunichar{251C}}
  \sphinxDUC{2572}{\textbackslash}
\fi
\usepackage{cmap}
\usepackage[T1]{fontenc}
\usepackage{amsmath,amssymb,amstext}
\usepackage{babel}



\usepackage{tgtermes}
\usepackage{tgheros}
\renewcommand{\ttdefault}{txtt}



\usepackage[Bjarne]{fncychap}
\usepackage{sphinx}

\fvset{fontsize=auto}
\usepackage{geometry}


% Include hyperref last.
\usepackage{hyperref}
% Fix anchor placement for figures with captions.
\usepackage{hypcap}% it must be loaded after hyperref.
% Set up styles of URL: it should be placed after hyperref.
\urlstyle{same}


\usepackage{sphinxmessages}




\title{KaneToolbox}
\date{Dec 20, 2023}
\release{1}
\author{Pietro Pustina}
\newcommand{\sphinxlogo}{\vbox{}}
\renewcommand{\releasename}{Release}
\makeindex
\begin{document}

\ifdefined\shorthandoff
  \ifnum\catcode`\=\string=\active\shorthandoff{=}\fi
  \ifnum\catcode`\"=\active\shorthandoff{"}\fi
\fi

\pagestyle{empty}
\sphinxmaketitle
\pagestyle{plain}
\sphinxtableofcontents
\pagestyle{normal}
\phantomsection\label{\detokenize{index::doc}}


\sphinxAtStartPar
\sphinxstylestrong{Kane Toolbox} is a MATLAB library to simulate mechanical systems of rigid and deformable bodies.

\begin{sphinxadmonition}{note}{Note:}
\sphinxAtStartPar
This project is under active development.
\end{sphinxadmonition}


\chapter{Contents}
\label{\detokenize{index:contents}}
\sphinxstepscope


\section{Usage}
\label{\detokenize{usage:usage}}\label{\detokenize{usage::doc}}

\subsection{Installation}
\label{\detokenize{usage:installation}}\label{\detokenize{usage:id1}}
\sphinxAtStartPar
Download the source code from the GitHub page

\begin{sphinxVerbatim}[commandchars=\\\{\}]
\PYG{g+gp+gpVirtualEnv}{(.venv)} \PYG{g+gp}{\PYGZdl{} }pip\PYG{+w}{ }install\PYG{+w}{ }lumache
\end{sphinxVerbatim}

\sphinxAtStartPar
and place it in your working directory.


\subsection{Initialization}
\label{\detokenize{usage:initialization}}
\sphinxAtStartPar
Add the library to the MATLAB path by running the \sphinxcode{\sphinxupquote{initToolbox.m}} script.

\begin{sphinxVerbatim}[commandchars=\\\{\}]
\PYG{n}{initToolbox}\PYG{p}{.}\PYG{n}{m}
\end{sphinxVerbatim}

\begin{sphinxadmonition}{note}{Note:}
\sphinxAtStartPar
The script has to be run every time a new instance of MATLAB is started.
\end{sphinxadmonition}

\sphinxAtStartPar
Check out the {\hyperref[\detokenize{usage::doc}]{\sphinxcrossref{\DUrole{doc}{Usage}}}} section for further information, including how to
{\hyperref[\detokenize{usage:installation}]{\sphinxcrossref{\DUrole{std,std-ref}{install}}}} the project.


\chapter{API}
\label{\detokenize{index:api}}
\sphinxstepscope


\section{API}
\label{\detokenize{api:api}}\label{\detokenize{api::doc}}
\sphinxstepscope


\subsection{BodyTree}
\label{\detokenize{bodytree:bodytree}}\label{\detokenize{bodytree::doc}}\index{classes (module)@\spxentry{classes}\spxextra{module}}\phantomsection\label{\detokenize{bodytree:module-classes}}
\sphinxAtStartPar
This class represents a mechanical system consisting of {\hyperref[\detokenize{body:Body}]{\sphinxcrossref{\sphinxcode{\sphinxupquote{bodies}}}}} and {\hyperref[\detokenize{joint:Joint}]{\sphinxcrossref{\sphinxcode{\sphinxupquote{joints}}}}} serially interconneted.
\index{BodyTree (built\sphinxhyphen{}in class)@\spxentry{BodyTree}\spxextra{built\sphinxhyphen{}in class}}

\begin{fulllineitems}
\phantomsection\label{\detokenize{bodytree:BodyTree}}
\pysigstartsignatures
\pysigline{\sphinxbfcode{\sphinxupquote{class }}\sphinxbfcode{\sphinxupquote{BodyTree}}}
\pysigstopsignatures
\sphinxAtStartPar
Bases: \sphinxcode{\sphinxupquote{handle}}

\sphinxAtStartPar
Class representing an open tree of joints and bodies serially connected.

\begin{sphinxuseclass}{members}\begin{description}
\sphinxlineitem{Constructor Summary}\index{BodyTree() (BodyTree method)@\spxentry{BodyTree()}\spxextra{BodyTree method}}

\begin{fulllineitems}
\phantomsection\label{\detokenize{bodytree:BodyTree.BodyTree}}
\pysigstartsignatures
\pysiglinewithargsret{\sphinxbfcode{\sphinxupquote{BodyTree}}}{\sphinxparam{Joints}\sphinxparamcomma \sphinxparam{Bodies}}{}
\pysigstopsignatures
\sphinxAtStartPar
Construct a BodyTree consisting of joints and
bodies.
\begin{quote}\begin{description}
\sphinxlineitem{Parameters}\begin{itemize}
\item {} 
\sphinxAtStartPar
\sphinxstyleliteralstrong{\sphinxupquote{Joints}} ({\hyperref[\detokenize{joint:Joint}]{\sphinxcrossref{\sphinxcode{\sphinxupquote{Joint}}}}}) \textendash{} Cell array of joints

\item {} 
\sphinxAtStartPar
\sphinxstyleliteralstrong{\sphinxupquote{Bodies}} ({\hyperref[\detokenize{body:Body}]{\sphinxcrossref{\sphinxcode{\sphinxupquote{Body}}}}}) \textendash{} Cell array of bodies

\end{itemize}

\end{description}\end{quote}

\end{fulllineitems}


\sphinxlineitem{Property Summary}\index{Bodies (BodyTree attribute)@\spxentry{Bodies}\spxextra{BodyTree attribute}}

\begin{fulllineitems}
\phantomsection\label{\detokenize{bodytree:BodyTree.Bodies}}
\pysigstartsignatures
\pysigline{\sphinxbfcode{\sphinxupquote{Bodies}}}
\pysigstopsignatures
\sphinxAtStartPar
Bodies of the tree

\end{fulllineitems}

\index{Joints (BodyTree attribute)@\spxentry{Joints}\spxextra{BodyTree attribute}}

\begin{fulllineitems}
\phantomsection\label{\detokenize{bodytree:BodyTree.Joints}}
\pysigstartsignatures
\pysigline{\sphinxbfcode{\sphinxupquote{Joints}}}
\pysigstopsignatures
\sphinxAtStartPar
Joints of the tree

\end{fulllineitems}

\index{MassConditionNumber (BodyTree attribute)@\spxentry{MassConditionNumber}\spxextra{BodyTree attribute}}

\begin{fulllineitems}
\phantomsection\label{\detokenize{bodytree:BodyTree.MassConditionNumber}}
\pysigstartsignatures
\pysigline{\sphinxbfcode{\sphinxupquote{MassConditionNumber}}}
\pysigstopsignatures
\sphinxAtStartPar
Mass condition numer

\end{fulllineitems}

\index{MaxBodiesNumber (BodyTree attribute)@\spxentry{MaxBodiesNumber}\spxextra{BodyTree attribute}}

\begin{fulllineitems}
\phantomsection\label{\detokenize{bodytree:BodyTree.MaxBodiesNumber}}
\pysigstartsignatures
\pysigline{\sphinxbfcode{\sphinxupquote{MaxBodiesNumber}}}
\pysigstopsignatures
\sphinxAtStartPar
Define the maximum number of bodies and joints (required for code generation)

\end{fulllineitems}

\index{N\_B (BodyTree attribute)@\spxentry{N\_B}\spxextra{BodyTree attribute}}

\begin{fulllineitems}
\phantomsection\label{\detokenize{bodytree:BodyTree.N_B}}
\pysigstartsignatures
\pysigline{\sphinxbfcode{\sphinxupquote{N\_B}}}
\pysigstopsignatures
\sphinxAtStartPar
Total number of bodies in the tree

\end{fulllineitems}

\index{T0 (BodyTree attribute)@\spxentry{T0}\spxextra{BodyTree attribute}}

\begin{fulllineitems}
\phantomsection\label{\detokenize{bodytree:BodyTree.T0}}
\pysigstartsignatures
\pysigline{\sphinxbfcode{\sphinxupquote{T0}}}
\pysigstopsignatures
\sphinxAtStartPar
Orientation of the base with respect to the world frame

\end{fulllineitems}

\index{g (BodyTree attribute)@\spxentry{g}\spxextra{BodyTree attribute}}

\begin{fulllineitems}
\phantomsection\label{\detokenize{bodytree:BodyTree.g}}
\pysigstartsignatures
\pysigline{\sphinxbfcode{\sphinxupquote{g}}}
\pysigstopsignatures
\sphinxAtStartPar
Gravity field in the base frame

\end{fulllineitems}

\index{n (BodyTree attribute)@\spxentry{n}\spxextra{BodyTree attribute}}

\begin{fulllineitems}
\phantomsection\label{\detokenize{bodytree:BodyTree.n}}
\pysigstartsignatures
\pysigline{\sphinxbfcode{\sphinxupquote{n}}}
\pysigstopsignatures
\sphinxAtStartPar
Total number of DoFs

\end{fulllineitems}


\sphinxlineitem{Method Summary}\index{ApparentForce() (BodyTree method)@\spxentry{ApparentForce()}\spxextra{BodyTree method}}

\begin{fulllineitems}
\phantomsection\label{\detokenize{bodytree:BodyTree.ApparentForce}}
\pysigstartsignatures
\pysiglinewithargsret{\sphinxbfcode{\sphinxupquote{ApparentForce}}}{\sphinxparam{q}\sphinxparamcomma \sphinxparam{dq}}{}
\pysigstopsignatures
\sphinxAtStartPar
Set gravity force to zero

\end{fulllineitems}

\index{ApparentMatrix() (BodyTree method)@\spxentry{ApparentMatrix()}\spxextra{BodyTree method}}

\begin{fulllineitems}
\phantomsection\label{\detokenize{bodytree:BodyTree.ApparentMatrix}}
\pysigstartsignatures
\pysiglinewithargsret{\sphinxbfcode{\sphinxupquote{ApparentMatrix}}}{\sphinxparam{q}\sphinxparamcomma \sphinxparam{dq}}{}
\pysigstopsignatures
\sphinxAtStartPar
Set gravity force to zero

\end{fulllineitems}

\index{D() (BodyTree method)@\spxentry{D()}\spxextra{BodyTree method}}

\begin{fulllineitems}
\phantomsection\label{\detokenize{bodytree:BodyTree.D}}
\pysigstartsignatures
\pysiglinewithargsret{\sphinxbfcode{\sphinxupquote{D}}}{\sphinxparam{q}\sphinxparamcomma \sphinxparam{dq}}{}
\pysigstopsignatures
\sphinxAtStartPar
Update the tree only if q and dq are passed as arguments

\end{fulllineitems}

\index{DirectKinematics() (BodyTree method)@\spxentry{DirectKinematics()}\spxextra{BodyTree method}}

\begin{fulllineitems}
\phantomsection\label{\detokenize{bodytree:BodyTree.DirectKinematics}}
\pysigstartsignatures
\pysiglinewithargsret{\sphinxbfcode{\sphinxupquote{DirectKinematics}}}{\sphinxparam{q}}{}
\pysigstopsignatures
\sphinxAtStartPar
Compute the direct kinematics for each body.
T is a cell array of transformation matrices from the base to the tip.

\end{fulllineitems}

\index{Energy() (BodyTree method)@\spxentry{Energy()}\spxextra{BodyTree method}}

\begin{fulllineitems}
\phantomsection\label{\detokenize{bodytree:BodyTree.Energy}}
\pysigstartsignatures
\pysiglinewithargsret{\sphinxbfcode{\sphinxupquote{Energy}}}{\sphinxparam{q}\sphinxparamcomma \sphinxparam{dq}\sphinxparamcomma \sphinxparam{q\_ref}}{}
\pysigstopsignatures
\sphinxAtStartPar
Compute the system energy.
q\_ref is the reference configuration for the computation of
the elastic and gravitational energy. If it is not provided as
argument, 0 is assumed.

\end{fulllineitems}

\index{EquilibriumConfiguration() (BodyTree method)@\spxentry{EquilibriumConfiguration()}\spxextra{BodyTree method}}

\begin{fulllineitems}
\phantomsection\label{\detokenize{bodytree:BodyTree.EquilibriumConfiguration}}
\pysigstartsignatures
\pysiglinewithargsret{\sphinxbfcode{\sphinxupquote{EquilibriumConfiguration}}}{\sphinxparam{q0}\sphinxparamcomma \sphinxparam{tau}}{}
\pysigstopsignatures
\sphinxAtStartPar
options = optimoptions(‘fmincon’, ‘FunctionTolerance’, 1e\sphinxhyphen{}4);
q\_eq = fsolve(@(q) obj.EquilibriumEquation(q, tau), q0, options);

\end{fulllineitems}

\index{ForwardDynamics() (BodyTree method)@\spxentry{ForwardDynamics()}\spxextra{BodyTree method}}

\begin{fulllineitems}
\phantomsection\label{\detokenize{bodytree:BodyTree.ForwardDynamics}}
\pysigstartsignatures
\pysiglinewithargsret{\sphinxbfcode{\sphinxupquote{ForwardDynamics}}}{\sphinxparam{q}\sphinxparamcomma \sphinxparam{dq}\sphinxparamcomma \sphinxparam{u}}{}
\pysigstopsignatures
\sphinxAtStartPar
Compute the mass matrix

\end{fulllineitems}

\index{InverseDynamics() (BodyTree method)@\spxentry{InverseDynamics()}\spxextra{BodyTree method}}

\begin{fulllineitems}
\phantomsection\label{\detokenize{bodytree:BodyTree.InverseDynamics}}
\pysigstartsignatures
\pysiglinewithargsret{\sphinxbfcode{\sphinxupquote{InverseDynamics}}}{\sphinxparam{q}\sphinxparamcomma \sphinxparam{dq}\sphinxparamcomma \sphinxparam{ddq}}{}
\pysigstopsignatures
\sphinxAtStartPar
Check that the vectors q, dq and ddq are columns.

\end{fulllineitems}

\index{K() (BodyTree method)@\spxentry{K()}\spxextra{BodyTree method}}

\begin{fulllineitems}
\phantomsection\label{\detokenize{bodytree:BodyTree.K}}
\pysigstartsignatures
\pysiglinewithargsret{\sphinxbfcode{\sphinxupquote{K}}}{\sphinxparam{q}}{}
\pysigstopsignatures
\sphinxAtStartPar
Update the tree only if q is passed as argument

\end{fulllineitems}

\index{Kane\_aux() (BodyTree method)@\spxentry{Kane\_aux()}\spxextra{BodyTree method}}

\begin{fulllineitems}
\phantomsection\label{\detokenize{bodytree:BodyTree.Kane_aux}}
\pysigstartsignatures
\pysiglinewithargsret{\sphinxbfcode{\sphinxupquote{Kane\_aux}}}{\sphinxparam{g}\sphinxparamcomma \sphinxparam{q\_type}}{}
\pysigstopsignatures
\sphinxAtStartPar
KANE\_AUX Given a obj model compute the generalized force that realizes ddq in
the state (q,dq)

\end{fulllineitems}

\index{MassMatrix() (BodyTree method)@\spxentry{MassMatrix()}\spxextra{BodyTree method}}

\begin{fulllineitems}
\phantomsection\label{\detokenize{bodytree:BodyTree.MassMatrix}}
\pysigstartsignatures
\pysiglinewithargsret{\sphinxbfcode{\sphinxupquote{MassMatrix}}}{\sphinxparam{q}}{}
\pysigstopsignatures
\sphinxAtStartPar
Store value of gravity and set gravity to zero to compute the
mass matrix

\end{fulllineitems}

\index{TreeUpdate() (BodyTree method)@\spxentry{TreeUpdate()}\spxextra{BodyTree method}}

\begin{fulllineitems}
\phantomsection\label{\detokenize{bodytree:BodyTree.TreeUpdate}}
\pysigstartsignatures
\pysiglinewithargsret{\sphinxbfcode{\sphinxupquote{TreeUpdate}}}{\sphinxparam{q}\sphinxparamcomma \sphinxparam{dq}\sphinxparamcomma \sphinxparam{ddq}}{}
\pysigstopsignatures
\sphinxAtStartPar
Update the state of the tree.

\end{fulllineitems}


\end{description}

\end{sphinxuseclass}
\end{fulllineitems}


\sphinxstepscope


\subsection{Body}
\label{\detokenize{body:body}}\label{\detokenize{body::doc}}\index{classes (module)@\spxentry{classes}\spxextra{module}}\phantomsection\label{\detokenize{body:module-classes}}
\sphinxAtStartPar
This abstract class represents a generic body. The class defines all the methods that any body must implement to be used by the methods {\hyperref[\detokenize{bodytree:BodyTree}]{\sphinxcrossref{\sphinxcode{\sphinxupquote{BodyTree}}}}}.
\index{Body (built\sphinxhyphen{}in class)@\spxentry{Body}\spxextra{built\sphinxhyphen{}in class}}

\begin{fulllineitems}
\phantomsection\label{\detokenize{body:Body}}
\pysigstartsignatures
\pysigline{\sphinxbfcode{\sphinxupquote{class }}\sphinxbfcode{\sphinxupquote{Body}}}
\pysigstopsignatures
\sphinxAtStartPar
Bases: \sphinxcode{\sphinxupquote{handle}}

\sphinxAtStartPar
Abstract class representsing a generic body of the kinematic tree.

\begin{sphinxuseclass}{members}\begin{description}
\sphinxlineitem{Property Summary}\index{Parameters (Body attribute)@\spxentry{Parameters}\spxextra{Body attribute}}

\begin{fulllineitems}
\phantomsection\label{\detokenize{body:Body.Parameters}}
\pysigstartsignatures
\pysigline{\sphinxbfcode{\sphinxupquote{Parameters}}}
\pysigstopsignatures
\sphinxAtStartPar
Vector of parameters of the body

\end{fulllineitems}

\index{n (Body attribute)@\spxentry{n}\spxextra{Body attribute}}

\begin{fulllineitems}
\phantomsection\label{\detokenize{body:Body.n}}
\pysigstartsignatures
\pysigline{\sphinxbfcode{\sphinxupquote{n}}}
\pysigstopsignatures
\sphinxAtStartPar
Number of degrees of freedom of the body

\end{fulllineitems}


\sphinxlineitem{Method Summary}\index{T() (Body method)@\spxentry{T()}\spxextra{Body method}}

\begin{fulllineitems}
\phantomsection\label{\detokenize{body:Body.T}}
\pysigstartsignatures
\pysiglinewithargsret{\sphinxbfcode{\sphinxupquote{T}}}{\sphinxparam{q}}{}
\pysigstopsignatures
\sphinxAtStartPar
Transformation matrix from the distal end of the body to
the base.
\begin{quote}\begin{description}
\sphinxlineitem{Parameters}
\sphinxAtStartPar
\sphinxstyleliteralstrong{\sphinxupquote{q}} (\sphinxcode{\sphinxupquote{double, symbolic}}) \textendash{} Configuration at which the transformation is evaluated

\sphinxlineitem{Returns}
\sphinxAtStartPar
4x4 homogeneous transformation matrix

\sphinxlineitem{Return type}
\sphinxAtStartPar
(double, symbolic)

\end{description}\end{quote}

\end{fulllineitems}

\index{Update() (Body method)@\spxentry{Update()}\spxextra{Body method}}

\begin{fulllineitems}
\phantomsection\label{\detokenize{body:Body.Update}}
\pysigstartsignatures
\pysiglinewithargsret{\sphinxbfcode{\sphinxupquote{Update}}}{\sphinxparam{q}\sphinxparamcomma \sphinxparam{dq}\sphinxparamcomma \sphinxparam{ddq}}{}
\pysigstopsignatures
\sphinxAtStartPar
Update the current state of the body.
\begin{quote}\begin{description}
\sphinxlineitem{Parameters}\begin{itemize}
\item {} 
\sphinxAtStartPar
\sphinxstyleliteralstrong{\sphinxupquote{q}} (\sphinxcode{\sphinxupquote{double, symbolic}}) \textendash{} Current configuration

\item {} 
\sphinxAtStartPar
\sphinxstyleliteralstrong{\sphinxupquote{dq}} (\sphinxcode{\sphinxupquote{double, symbolic}}) \textendash{} First time derivative of q

\item {} 
\sphinxAtStartPar
\sphinxstyleliteralstrong{\sphinxupquote{ddq}} (\sphinxcode{\sphinxupquote{double, symbolic}}) \textendash{} Second time derivative of q

\end{itemize}

\end{description}\end{quote}

\end{fulllineitems}

\index{grad\_int\_dr() (Body method)@\spxentry{grad\_int\_dr()}\spxextra{Body method}}

\begin{fulllineitems}
\phantomsection\label{\detokenize{body:Body.grad_int_dr}}
\pysigstartsignatures
\pysiglinewithargsret{\sphinxbfcode{\sphinxupquote{grad\_int\_dr}}}{\sphinxparam{q}}{}
\pysigstopsignatures
\sphinxAtStartPar
Todo: This is always zero, remove.

\end{fulllineitems}

\index{int\_ddr() (Body method)@\spxentry{int\_ddr()}\spxextra{Body method}}

\begin{fulllineitems}
\phantomsection\label{\detokenize{body:Body.int_ddr}}
\pysigstartsignatures
\pysiglinewithargsret{\sphinxbfcode{\sphinxupquote{int\_ddr}}}{\sphinxparam{\textasciitilde{}}\sphinxparamcomma \sphinxparam{q}\sphinxparamcomma \sphinxparam{\textasciitilde{}}\sphinxparamcomma \sphinxparam{\textasciitilde{}}}{}
\pysigstopsignatures
\sphinxAtStartPar
Todo: This is always zero, remove.

\end{fulllineitems}

\index{int\_dr() (Body method)@\spxentry{int\_dr()}\spxextra{Body method}}

\begin{fulllineitems}
\phantomsection\label{\detokenize{body:Body.int_dr}}
\pysigstartsignatures
\pysiglinewithargsret{\sphinxbfcode{\sphinxupquote{int\_dr}}}{\sphinxparam{\textasciitilde{}}\sphinxparamcomma \sphinxparam{q}\sphinxparamcomma \sphinxparam{\textasciitilde{}}}{}
\pysigstopsignatures
\sphinxAtStartPar
Todo: This is always zero, remove.

\end{fulllineitems}

\index{int\_dr\_O\_dr() (Body method)@\spxentry{int\_dr\_O\_dr()}\spxextra{Body method}}

\begin{fulllineitems}
\phantomsection\label{\detokenize{body:Body.int_dr_O_dr}}
\pysigstartsignatures
\pysiglinewithargsret{\sphinxbfcode{\sphinxupquote{int\_dr\_O\_dr}}}{\sphinxparam{\textasciitilde{}}\sphinxparamcomma \sphinxparam{\textasciitilde{}}\sphinxparamcomma \sphinxparam{\textasciitilde{}}}{}
\pysigstopsignatures
\sphinxAtStartPar
Todo: This is always zero, remove.

\end{fulllineitems}

\index{toStruct() (Body method)@\spxentry{toStruct()}\spxextra{Body method}}

\begin{fulllineitems}
\phantomsection\label{\detokenize{body:Body.toStruct}}
\pysigstartsignatures
\pysiglinewithargsret{\sphinxbfcode{\sphinxupquote{toStruct}}}{}{}
\pysigstopsignatures
\sphinxAtStartPar
Convert object to a struct representation.

\end{fulllineitems}


\end{description}

\end{sphinxuseclass}
\end{fulllineitems}


\sphinxstepscope


\subsubsection{GVSBody}
\label{\detokenize{gvsbody:gvsbody}}\label{\detokenize{gvsbody::doc}}\index{classes (module)@\spxentry{classes}\spxextra{module}}\phantomsection\label{\detokenize{gvsbody:module-classes}}
\sphinxAtStartPar
This abstract class represents a slender soft body modeled under the Geometric Variable Strain (GVS) hypothesis.
The class implements all the methods required by a {\hyperref[\detokenize{body:Body}]{\sphinxcrossref{\sphinxcode{\sphinxupquote{Body}}}}}, but requires the definition of a strain function \(\xi(q, s)\), where \(q\) is the vector of configuration variables
and \(s \in [0, L]\) the arc length, with \(L\) the rest length of the body.

\sphinxAtStartPar
See {\hyperref[\detokenize{PCC/pcc2d:PCC2D}]{\sphinxcrossref{\sphinxcode{\sphinxupquote{PCC2D}}}}} or {\hyperref[\detokenize{PCC/pcc3d:PCC3D}]{\sphinxcrossref{\sphinxcode{\sphinxupquote{PCC3D}}}}} for possible implementations.
\index{GVSBody (built\sphinxhyphen{}in class)@\spxentry{GVSBody}\spxextra{built\sphinxhyphen{}in class}}

\begin{fulllineitems}
\phantomsection\label{\detokenize{gvsbody:GVSBody}}
\pysigstartsignatures
\pysigline{\sphinxbfcode{\sphinxupquote{class }}\sphinxbfcode{\sphinxupquote{GVSBody}}}
\pysigstopsignatures
\sphinxAtStartPar
Bases: {\hyperref[\detokenize{body:Body}]{\sphinxcrossref{\sphinxcode{\sphinxupquote{Body}}}}}

\sphinxAtStartPar
GVSBODY Class representing a slender body modeled under the geometric
variable strain approach.

\begin{sphinxuseclass}{members}\begin{description}
\sphinxlineitem{Property Summary}\index{Parameters (GVSBody attribute)@\spxentry{Parameters}\spxextra{GVSBody attribute}}

\begin{fulllineitems}
\phantomsection\label{\detokenize{gvsbody:GVSBody.Parameters}}
\pysigstartsignatures
\pysigline{\sphinxbfcode{\sphinxupquote{Parameters}}}
\pysigstopsignatures
\sphinxAtStartPar
Parameters are in this order:
\sphinxhyphen{}Rest length
\sphinxhyphen{}Base radius
\sphinxhyphen{}Tip radius
\sphinxhyphen{}Mass density
\sphinxhyphen{}Young modulus
\sphinxhyphen{}Poisson ratio
\sphinxhyphen{}Material damping
\sphinxhyphen{}Number of Gaussian Points

\end{fulllineitems}

\index{ReferenceStrain (GVSBody attribute)@\spxentry{ReferenceStrain}\spxextra{GVSBody attribute}}

\begin{fulllineitems}
\phantomsection\label{\detokenize{gvsbody:GVSBody.ReferenceStrain}}
\pysigstartsignatures
\pysigline{\sphinxbfcode{\sphinxupquote{ReferenceStrain}}}
\pysigstopsignatures
\sphinxAtStartPar
Reference strain

\end{fulllineitems}


\sphinxlineitem{Method Summary}\index{D() (GVSBody method)@\spxentry{D()}\spxextra{GVSBody method}}

\begin{fulllineitems}
\phantomsection\label{\detokenize{gvsbody:GVSBody.D}}
\pysigstartsignatures
\pysiglinewithargsret{\sphinxbfcode{\sphinxupquote{D}}}{\sphinxparam{q}\sphinxparamcomma \sphinxparam{dq}}{}
\pysigstopsignatures
\sphinxAtStartPar
Variables initialization

\end{fulllineitems}

\index{I() (GVSBody method)@\spxentry{I()}\spxextra{GVSBody method}}

\begin{fulllineitems}
\phantomsection\label{\detokenize{gvsbody:GVSBody.I}}
\pysigstartsignatures
\pysiglinewithargsret{\sphinxbfcode{\sphinxupquote{I}}}{\sphinxparam{q}}{}
\pysigstopsignatures
\sphinxAtStartPar
Update the DK

\end{fulllineitems}

\index{J() (GVSBody method)@\spxentry{J()}\spxextra{GVSBody method}}

\begin{fulllineitems}
\phantomsection\label{\detokenize{gvsbody:GVSBody.J}}
\pysigstartsignatures
\pysiglinewithargsret{\sphinxbfcode{\sphinxupquote{J}}}{\sphinxparam{q}\sphinxparamcomma \sphinxparam{dq}}{}
\pysigstopsignatures
\sphinxAtStartPar
Update the kinematics

\end{fulllineitems}

\index{K() (GVSBody method)@\spxentry{K()}\spxextra{GVSBody method}}

\begin{fulllineitems}
\phantomsection\label{\detokenize{gvsbody:GVSBody.K}}
\pysigstartsignatures
\pysiglinewithargsret{\sphinxbfcode{\sphinxupquote{K}}}{\sphinxparam{q}}{}
\pysigstopsignatures
\sphinxAtStartPar
Variables initialization

\end{fulllineitems}

\index{Update() (GVSBody method)@\spxentry{Update()}\spxextra{GVSBody method}}

\begin{fulllineitems}
\phantomsection\label{\detokenize{gvsbody:GVSBody.Update}}
\pysigstartsignatures
\pysiglinewithargsret{\sphinxbfcode{\sphinxupquote{Update}}}{\sphinxparam{q}\sphinxparamcomma \sphinxparam{dq}\sphinxparamcomma \sphinxparam{ddq}}{}
\pysigstopsignatures
\sphinxAtStartPar
Update the kinematics

\end{fulllineitems}

\index{a\_com\_rel() (GVSBody method)@\spxentry{a\_com\_rel()}\spxextra{GVSBody method}}

\begin{fulllineitems}
\phantomsection\label{\detokenize{gvsbody:GVSBody.a_com_rel}}
\pysigstartsignatures
\pysiglinewithargsret{\sphinxbfcode{\sphinxupquote{a\_com\_rel}}}{\sphinxparam{q}\sphinxparamcomma \sphinxparam{dq}\sphinxparamcomma \sphinxparam{ddq}}{}
\pysigstopsignatures
\sphinxAtStartPar
Update the kinematics

\end{fulllineitems}

\index{grad\_J() (GVSBody method)@\spxentry{grad\_J()}\spxextra{GVSBody method}}

\begin{fulllineitems}
\phantomsection\label{\detokenize{gvsbody:GVSBody.grad_J}}
\pysigstartsignatures
\pysiglinewithargsret{\sphinxbfcode{\sphinxupquote{grad\_J}}}{\sphinxparam{q}}{}
\pysigstopsignatures
\sphinxAtStartPar
Update the DK

\end{fulllineitems}

\index{grad\_int\_dr() (GVSBody method)@\spxentry{grad\_int\_dr()}\spxextra{GVSBody method}}

\begin{fulllineitems}
\phantomsection\label{\detokenize{gvsbody:GVSBody.grad_int_dr}}
\pysigstartsignatures
\pysiglinewithargsret{\sphinxbfcode{\sphinxupquote{grad\_int\_dr}}}{\sphinxparam{q}}{}
\pysigstopsignatures
\sphinxAtStartPar
TODO: This is always zero, remove

\end{fulllineitems}

\index{grad\_int\_r\_X\_dr() (GVSBody method)@\spxentry{grad\_int\_r\_X\_dr()}\spxextra{GVSBody method}}

\begin{fulllineitems}
\phantomsection\label{\detokenize{gvsbody:GVSBody.grad_int_r_X_dr}}
\pysigstartsignatures
\pysiglinewithargsret{\sphinxbfcode{\sphinxupquote{grad\_int\_r\_X\_dr}}}{\sphinxparam{q}}{}
\pysigstopsignatures
\sphinxAtStartPar
Update the DK

\end{fulllineitems}

\index{grad\_v\_com() (GVSBody method)@\spxentry{grad\_v\_com()}\spxextra{GVSBody method}}

\begin{fulllineitems}
\phantomsection\label{\detokenize{gvsbody:GVSBody.grad_v_com}}
\pysigstartsignatures
\pysiglinewithargsret{\sphinxbfcode{\sphinxupquote{grad\_v\_com}}}{\sphinxparam{q}}{}
\pysigstopsignatures
\sphinxAtStartPar
Update the DK

\end{fulllineitems}

\index{int\_ddr() (GVSBody method)@\spxentry{int\_ddr()}\spxextra{GVSBody method}}

\begin{fulllineitems}
\phantomsection\label{\detokenize{gvsbody:GVSBody.int_ddr}}
\pysigstartsignatures
\pysiglinewithargsret{\sphinxbfcode{\sphinxupquote{int\_ddr}}}{\sphinxparam{q}\sphinxparamcomma \sphinxparam{dq}\sphinxparamcomma \sphinxparam{ddq}}{}
\pysigstopsignatures
\sphinxAtStartPar
TODO: This is always zero, remove

\end{fulllineitems}

\index{int\_dr() (GVSBody method)@\spxentry{int\_dr()}\spxextra{GVSBody method}}

\begin{fulllineitems}
\phantomsection\label{\detokenize{gvsbody:GVSBody.int_dr}}
\pysigstartsignatures
\pysiglinewithargsret{\sphinxbfcode{\sphinxupquote{int\_dr}}}{\sphinxparam{q}\sphinxparamcomma \sphinxparam{dq}}{}
\pysigstopsignatures
\sphinxAtStartPar
TODO: This is always zero, remove

\end{fulllineitems}

\index{int\_dr\_O\_dr() (GVSBody method)@\spxentry{int\_dr\_O\_dr()}\spxextra{GVSBody method}}

\begin{fulllineitems}
\phantomsection\label{\detokenize{gvsbody:GVSBody.int_dr_O_dr}}
\pysigstartsignatures
\pysiglinewithargsret{\sphinxbfcode{\sphinxupquote{int\_dr\_O\_dr}}}{\sphinxparam{q}\sphinxparamcomma \sphinxparam{dq}}{}
\pysigstopsignatures
\sphinxAtStartPar
TODO: This is always zero, remove

\end{fulllineitems}

\index{int\_dr\_X\_pv\_r() (GVSBody method)@\spxentry{int\_dr\_X\_pv\_r()}\spxextra{GVSBody method}}

\begin{fulllineitems}
\phantomsection\label{\detokenize{gvsbody:GVSBody.int_dr_X_pv_r}}
\pysigstartsignatures
\pysiglinewithargsret{\sphinxbfcode{\sphinxupquote{int\_dr\_X\_pv\_r}}}{\sphinxparam{q}\sphinxparamcomma \sphinxparam{dq}}{}
\pysigstopsignatures
\sphinxAtStartPar
Update the kinematics

\end{fulllineitems}

\index{int\_pv\_r\_O\_dd\_r() (GVSBody method)@\spxentry{int\_pv\_r\_O\_dd\_r()}\spxextra{GVSBody method}}

\begin{fulllineitems}
\phantomsection\label{\detokenize{gvsbody:GVSBody.int_pv_r_O_dd_r}}
\pysigstartsignatures
\pysiglinewithargsret{\sphinxbfcode{\sphinxupquote{int\_pv\_r\_O\_dd\_r}}}{\sphinxparam{q}\sphinxparamcomma \sphinxparam{dq}\sphinxparamcomma \sphinxparam{ddq}}{}
\pysigstopsignatures
\sphinxAtStartPar
Update the kinematics

\end{fulllineitems}

\index{int\_r\_X\_ddr() (GVSBody method)@\spxentry{int\_r\_X\_ddr()}\spxextra{GVSBody method}}

\begin{fulllineitems}
\phantomsection\label{\detokenize{gvsbody:GVSBody.int_r_X_ddr}}
\pysigstartsignatures
\pysiglinewithargsret{\sphinxbfcode{\sphinxupquote{int\_r\_X\_ddr}}}{\sphinxparam{q}\sphinxparamcomma \sphinxparam{dq}\sphinxparamcomma \sphinxparam{ddq}}{}
\pysigstopsignatures
\sphinxAtStartPar
Update the kinematics

\end{fulllineitems}

\index{int\_r\_X\_dr() (GVSBody method)@\spxentry{int\_r\_X\_dr()}\spxextra{GVSBody method}}

\begin{fulllineitems}
\phantomsection\label{\detokenize{gvsbody:GVSBody.int_r_X_dr}}
\pysigstartsignatures
\pysiglinewithargsret{\sphinxbfcode{\sphinxupquote{int\_r\_X\_dr}}}{\sphinxparam{q}\sphinxparamcomma \sphinxparam{dq}}{}
\pysigstopsignatures
\sphinxAtStartPar
Update the kinematics

\end{fulllineitems}

\index{p\_com() (GVSBody method)@\spxentry{p\_com()}\spxextra{GVSBody method}}

\begin{fulllineitems}
\phantomsection\label{\detokenize{gvsbody:GVSBody.p_com}}
\pysigstartsignatures
\pysiglinewithargsret{\sphinxbfcode{\sphinxupquote{p\_com}}}{\sphinxparam{q}}{}
\pysigstopsignatures
\sphinxAtStartPar
Update the DK

\end{fulllineitems}

\index{v\_com\_rel() (GVSBody method)@\spxentry{v\_com\_rel()}\spxextra{GVSBody method}}

\begin{fulllineitems}
\phantomsection\label{\detokenize{gvsbody:GVSBody.v_com_rel}}
\pysigstartsignatures
\pysiglinewithargsret{\sphinxbfcode{\sphinxupquote{v\_com\_rel}}}{\sphinxparam{q}\sphinxparamcomma \sphinxparam{dq}}{}
\pysigstopsignatures
\sphinxAtStartPar
Update the kinematics

\end{fulllineitems}


\end{description}

\end{sphinxuseclass}
\end{fulllineitems}


\sphinxstepscope


\paragraph{PCC3D}
\label{\detokenize{PCC/pcc3d:pcc3d}}\label{\detokenize{PCC/pcc3d::doc}}\index{classes (module)@\spxentry{classes}\spxextra{module}}\phantomsection\label{\detokenize{PCC/pcc3d:module-classes}}
\sphinxAtStartPar
Class representing a slender soft body modeled under the piecewise constant curvature hypothesis with elongation.
The strain is modeled as
\begin{equation*}
\begin{split}\xi(q, s) = \left(\begin{array}{c} \displaystyle \frac{q_{2}}{L} \\ \displaystyle -\frac{q_{1}}{L} \\ 0 \\ 0 \\ 0 \\ \displaystyle \frac{q_{3} + L}{L}  \end{array}\right),\end{split}
\end{equation*}
\sphinxAtStartPar
where \(L\) is the rest length of the body.
\index{PCC3D (built\sphinxhyphen{}in class)@\spxentry{PCC3D}\spxextra{built\sphinxhyphen{}in class}}

\begin{fulllineitems}
\phantomsection\label{\detokenize{PCC/pcc3d:PCC3D}}
\pysigstartsignatures
\pysigline{\sphinxbfcode{\sphinxupquote{class }}\sphinxbfcode{\sphinxupquote{PCC3D}}}
\pysigstopsignatures
\sphinxAtStartPar
Bases: {\hyperref[\detokenize{gvsbody:GVSBody}]{\sphinxcrossref{\sphinxcode{\sphinxupquote{GVSBody}}}}}

\sphinxAtStartPar
GVSBODY Class representing a slender 3D body modeled under the PCC
hypothesis.

\end{fulllineitems}


\sphinxstepscope


\paragraph{PCC2D}
\label{\detokenize{PCC/pcc2d:pcc2d}}\label{\detokenize{PCC/pcc2d::doc}}\index{classes (module)@\spxentry{classes}\spxextra{module}}\phantomsection\label{\detokenize{PCC/pcc2d:module-classes}}
\sphinxAtStartPar
Class representing a slender soft body modeled under the piecewise constant curvature hypothesis without elongation. The strain is modeled as
\begin{equation*}
\begin{split}\xi(q, s) = \left(\begin{array}{c} 0 \\ \displaystyle -\frac{q}{L} \\ 0 \\ 0 \\ 0 \\ 1  \end{array}\right),\end{split}
\end{equation*}
\sphinxAtStartPar
where \(L\) is the rest length of the body.
\index{PCC2D (built\sphinxhyphen{}in class)@\spxentry{PCC2D}\spxextra{built\sphinxhyphen{}in class}}

\begin{fulllineitems}
\phantomsection\label{\detokenize{PCC/pcc2d:PCC2D}}
\pysigstartsignatures
\pysigline{\sphinxbfcode{\sphinxupquote{class }}\sphinxbfcode{\sphinxupquote{PCC2D}}}
\pysigstopsignatures
\sphinxAtStartPar
Bases: {\hyperref[\detokenize{gvsbody:GVSBody}]{\sphinxcrossref{\sphinxcode{\sphinxupquote{GVSBody}}}}}

\sphinxAtStartPar
GVSBODY Class representing a slender body modeled under the PCC
hypothesis without elongation.

\end{fulllineitems}


\sphinxstepscope


\subsubsection{RigidBody}
\label{\detokenize{rigidbody:rigidbody}}\label{\detokenize{rigidbody::doc}}\index{classes (module)@\spxentry{classes}\spxextra{module}}\phantomsection\label{\detokenize{rigidbody:module-classes}}
\sphinxAtStartPar
This abstract class represents a rigid body. The class implements all the methods required by a {\hyperref[\detokenize{body:Body}]{\sphinxcrossref{\sphinxcode{\sphinxupquote{Body}}}}}.
\index{RigidBody (built\sphinxhyphen{}in class)@\spxentry{RigidBody}\spxextra{built\sphinxhyphen{}in class}}

\begin{fulllineitems}
\phantomsection\label{\detokenize{rigidbody:RigidBody}}
\pysigstartsignatures
\pysigline{\sphinxbfcode{\sphinxupquote{class }}\sphinxbfcode{\sphinxupquote{RigidBody}}}
\pysigstopsignatures
\sphinxAtStartPar
Bases: {\hyperref[\detokenize{body:Body}]{\sphinxcrossref{\sphinxcode{\sphinxupquote{Body}}}}}

\sphinxAtStartPar
RIGIDBODY Class representing a rigid body. A rigid body is regarded as
a body with zero degrees of freedom. To represent a floating rigid
body attach a 6 degrees of freedom joint to the body.

\begin{sphinxuseclass}{members}\begin{description}
\sphinxlineitem{Property Summary}\index{Parameters (RigidBody attribute)@\spxentry{Parameters}\spxextra{RigidBody attribute}}

\begin{fulllineitems}
\phantomsection\label{\detokenize{rigidbody:RigidBody.Parameters}}
\pysigstartsignatures
\pysigline{\sphinxbfcode{\sphinxupquote{Parameters}}}
\pysigstopsignatures
\sphinxAtStartPar
A rigid body has 10 parameters organized as follows
Parameters = {[}Mass; P\_com\_x; P\_com\_y; P\_com\_z; I\_xx; I\_yy; I\_zz; I\_xy; I\_xz; I\_yz{]};

\end{fulllineitems}

\index{n (RigidBody attribute)@\spxentry{n}\spxextra{RigidBody attribute}}

\begin{fulllineitems}
\phantomsection\label{\detokenize{rigidbody:RigidBody.n}}
\pysigstartsignatures
\pysigline{\sphinxbfcode{\sphinxupquote{n}}}
\pysigstopsignatures
\sphinxAtStartPar
Number of degrees of freedom of the body.

\end{fulllineitems}


\end{description}

\end{sphinxuseclass}
\end{fulllineitems}


\sphinxstepscope


\subsection{Joint}
\label{\detokenize{joint:joint}}\label{\detokenize{joint::doc}}\index{classes (module)@\spxentry{classes}\spxextra{module}}\phantomsection\label{\detokenize{joint:module-classes}}
\sphinxAtStartPar
This abstract class represents a generic joint. The class defines all the methods that any joint must implement to be used by the methods of {\hyperref[\detokenize{bodytree:BodyTree}]{\sphinxcrossref{\sphinxcode{\sphinxupquote{BodyTree}}}}}.
\index{Joint (built\sphinxhyphen{}in class)@\spxentry{Joint}\spxextra{built\sphinxhyphen{}in class}}

\begin{fulllineitems}
\phantomsection\label{\detokenize{joint:Joint}}
\pysigstartsignatures
\pysigline{\sphinxbfcode{\sphinxupquote{class }}\sphinxbfcode{\sphinxupquote{Joint}}}
\pysigstopsignatures
\sphinxAtStartPar
Bases: {\hyperref[\detokenize{body:Body}]{\sphinxcrossref{\sphinxcode{\sphinxupquote{Body}}}}}

\sphinxAtStartPar
Class that represents a generic joint of the kinematic tree.

\begin{sphinxuseclass}{members}\begin{description}
\sphinxlineitem{Method Summary}\index{toStruct() (Joint method)@\spxentry{toStruct()}\spxextra{Joint method}}

\begin{fulllineitems}
\phantomsection\label{\detokenize{joint:Joint.toStruct}}
\pysigstartsignatures
\pysiglinewithargsret{\sphinxbfcode{\sphinxupquote{toStruct}}}{}{}
\pysigstopsignatures
\sphinxAtStartPar
Convert object to a struct representation.

\end{fulllineitems}


\end{description}

\end{sphinxuseclass}
\end{fulllineitems}


\sphinxstepscope


\subsubsection{FixedJoint}
\label{\detokenize{fixedjoint:fixedjoint}}\label{\detokenize{fixedjoint::doc}}\index{classes (module)@\spxentry{classes}\spxextra{module}}\phantomsection\label{\detokenize{fixedjoint:module-classes}}
\sphinxAtStartPar
Class representing a fixed joint (hinge) between two bodies.
\index{FixedJoint (built\sphinxhyphen{}in class)@\spxentry{FixedJoint}\spxextra{built\sphinxhyphen{}in class}}

\begin{fulllineitems}
\phantomsection\label{\detokenize{fixedjoint:FixedJoint}}
\pysigstartsignatures
\pysigline{\sphinxbfcode{\sphinxupquote{class }}\sphinxbfcode{\sphinxupquote{FixedJoint}}}
\pysigstopsignatures
\sphinxAtStartPar
Bases: {\hyperref[\detokenize{joint:Joint}]{\sphinxcrossref{\sphinxcode{\sphinxupquote{Joint}}}}}

\sphinxAtStartPar
FIXEDJOINT Implements a fixed joint between two bodies.

\begin{sphinxuseclass}{members}\begin{description}
\sphinxlineitem{Property Summary}\index{n (FixedJoint attribute)@\spxentry{n}\spxextra{FixedJoint attribute}}

\begin{fulllineitems}
\phantomsection\label{\detokenize{fixedjoint:FixedJoint.n}}
\pysigstartsignatures
\pysigline{\sphinxbfcode{\sphinxupquote{n}}}
\pysigstopsignatures
\sphinxAtStartPar
Number of degrees of freedom of the joint

\end{fulllineitems}


\end{description}

\end{sphinxuseclass}
\end{fulllineitems}


\sphinxstepscope


\subsubsection{RotationalJoint}
\label{\detokenize{rotationaljoint:rotationaljoint}}\label{\detokenize{rotationaljoint::doc}}\index{classes (module)@\spxentry{classes}\spxextra{module}}\phantomsection\label{\detokenize{rotationaljoint:module-classes}}
\sphinxAtStartPar
Class representing a 1 degrees of freedom rotational joint.
\index{RotationalJoint (built\sphinxhyphen{}in class)@\spxentry{RotationalJoint}\spxextra{built\sphinxhyphen{}in class}}

\begin{fulllineitems}
\phantomsection\label{\detokenize{rotationaljoint:RotationalJoint}}
\pysigstartsignatures
\pysigline{\sphinxbfcode{\sphinxupquote{class }}\sphinxbfcode{\sphinxupquote{RotationalJoint}}}
\pysigstopsignatures
\sphinxAtStartPar
Bases: {\hyperref[\detokenize{joint:Joint}]{\sphinxcrossref{\sphinxcode{\sphinxupquote{Joint}}}}}

\sphinxAtStartPar
ROTATIONALJOINT Implements a rotational joint between two bodies.

\begin{sphinxuseclass}{members}\begin{description}
\sphinxlineitem{Property Summary}\index{Parameters (RotationalJoint attribute)@\spxentry{Parameters}\spxextra{RotationalJoint attribute}}

\begin{fulllineitems}
\phantomsection\label{\detokenize{rotationaljoint:RotationalJoint.Parameters}}
\pysigstartsignatures
\pysigline{\sphinxbfcode{\sphinxupquote{Parameters}}}
\pysigstopsignatures
\sphinxAtStartPar
The parameters of the joint are its DH parameters organized as
follows: Parameters = {[}alpha; a; d; theta{]}

\end{fulllineitems}

\index{n (RotationalJoint attribute)@\spxentry{n}\spxextra{RotationalJoint attribute}}

\begin{fulllineitems}
\phantomsection\label{\detokenize{rotationaljoint:RotationalJoint.n}}
\pysigstartsignatures
\pysigline{\sphinxbfcode{\sphinxupquote{n}}}
\pysigstopsignatures
\sphinxAtStartPar
Number of degrees of freedom of the joint

\end{fulllineitems}


\end{description}

\end{sphinxuseclass}
\end{fulllineitems}



\renewcommand{\indexname}{MATLAB Module Index}
\begin{sphinxtheindex}
\let\bigletter\sphinxstyleindexlettergroup
\bigletter{c}
\item\relax\sphinxstyleindexentry{classes}\sphinxstyleindexpageref{rotationaljoint:\detokenize{module-classes}}
\end{sphinxtheindex}

\renewcommand{\indexname}{Index}
\printindex
\end{document}